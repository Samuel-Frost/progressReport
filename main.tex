\documentclass[10pt,a4paper,twocolumn,twoside]{extarticle}

\title{Progress Report}
\author{Samuel James Frost}
\def\ID{20184093}
\input{header.tex}

\newcommand{\kcal}{kcal mol\(^{-1}\)}
\usepackage{amsmath}
\usepackage[style=chem-rsc,citestyle=chem-rsc]{biblatex}
\addbibresource{references.bib}

\usepackage[font=small]{caption}
\usepackage{lipsum}
\usepackage{xcolor}

\newcommand{\al}{\emph{et al. }}
\newcommand{\oA}{\si{\angstrom}}
\renewcommand{\d}{\text{d}}
\title{Frist Year Progress Report}
\author{Samuel~J.~Frost}

\begin{document}
	\thispagestyle{empty}
	\twocolumn[
	\begin{@twocolumnfalse}
		\begin{center}
			\vspace*{-10mm}
			{\Large\scshape\papertitle}\\
			\vspace{2ex}
			{\itshape\paperauthor}
		\end{center}
		\centering\noindent\rule{0.9\textwidth}{0.4pt}
		\begin{abstract}
			Certainly! Here's an abstract for your first year PhD progress report!
		\end{abstract}
		\centering\noindent\rule{0.9\textwidth}{0.4pt}\\
		\vspace{1cm}
	\end{@twocolumnfalse}]
	\pagenumbering{arabic}
	\tableofcontents

	
\section{Introduction}
The final aim of my PhD is to be able to accurately model the effects of radiation damage in diamond over large time scales, taking in to account quantum mechanical effects.

\section{Qunatum Tunnelling of Hydrogen in N$_2$VH}


\section{Vacancy Migration in a Pure Diamond Lattice}

There is much concern of how to remove vacancies from diamond, in order to create a perfect crystal. Annealing techniques are used to heat the diamond up to a certain temperature and cool it down, in hopes of making the vacancies turn mobile and rising to the surface, effectively eliminating themselves from the bulk. Previous experimental results have shown that this starts to occur at ~$800$\,K (CITATION NEEDED, NUMBER COULD BE WRONG), however it could also be useful to know the energies required for a vacancy to migrate, and also the energy barrier involved for the formation of a vacancy. Molecular dynamics simulations can provide an insight into how the mechanisms involved in vacancy migration, and to inform later experiments. Hu \al, as shown in section \ref{Hu}, have used a tersoff potential to model the vacancy migration of an atom from the second layer of a diamond ($001$) surface to the top layer. Their results showed that this process beigns to occur at $1400$\,K, deviating from known experimental results. Hu \al argue that is due to how the temperature of the diamond is measured in experiments, that the temperature of the surface is in fact much higher than measured, and so would align more with their findings. The results of Hu \al are quite old, however the methods used are not out of date, so it is reasonable to try and recreate similar results, and expand upon their work using more state-of-the-art techniques. 

The same system was set up as Hu \al described in section \ref{Hu}: a $1$\,fs timestep was used, as well as a $100$\,fs timestep for the velocity rescaling method, and a $1\,000$\,fs timestep for the pressure rescaling method. There is no mention in the original paper what the timestep for the velocity rescaling method is, nor is a pressure rescaling ensemble used at all. The choice to use the NPT ensemble came from the need to ensure that the crystal can expand slightly under higher temperatures. Following Hu \al, an initial system was created and relaxed at $300$\,K for $5$\,ps, before a vacancy was created in the second layer and it was allowed to relax again for another $5$\,ps. The final configuration of this system formed the starting configuration of all further simulations. The system was then allowed to run at temperatures in the $300 - 2\,000$\,K range. Taking inspiration from Hu \al, the positions of the atoms neighbouring the vacancy were extracted from the MD run. The atom with the lowest average distance to the vacancy site was taken to be the atom that was diffusing, which was then used for analysis. 

%Basically 1600 is when it really starts to migrate properly, we see two-step (maybe even three step) migration at 1600 and above, it does happen at 1400 but not always, probably a high chance of randomness -> monte carlo, at 2000 strange things happen because it is so hot, tersoff may be "breaking down". For GAP it seems to migrate at EVERY temperature, not good, we also see no migration from the third at any temperature for gap, also not good, could just be (001) surface however . Maybe explore NVT instead of NPT properly? -> makes no difference THEY USE CM^2 / s -> CHECK UNITS AND CONVERT


\section{Key Text Review}
\subsection{Hu \al}
\label{Hu}
"The Diffusion of Vacancies Near a Diamond ($001$) Surface" by Hu \al. has played an important role in my research, influencing a large part of section . Hu \al used molecular dynamics to investigate vacancy diffusion in diamond surfaces at various temperatures, calculating the diffusion coefficient and diffusion barrier. Knowing the properties of vacancy defects, and at what temperatures they are mobile such that they might escape to the surface is important when dealing with synthetic diamodns. The paper is limited in that it only deals with vacancies found in the second layer of the ($001$) surface. Other surfaces, such as the cleavage ($111$) surface, also play important roles in experiment, not to mention vacancies that are found further into the bulk, such as in the third or fourth layer. The limited scope of this paper has influenced my further research found in section .

They construct their simulation as a unit cell repeated equally 5 times in all 3 cartesian direction, with periodic boundary conditions along the $x$ and $y$ axes, and the surfaces in the $z$ direction showing a ($001$) face. There is no mention of the boundaries of the cell in the $z$ direction, however as it is dealing with a surface diffusion, it is reasonable to assume that there is a sufficent vacuum gap, such that there are no external forces acting on the surface. The perfect diamond crystal is first allowed to relax for $5$\,ps at $300$\,K, before having an atom in the second layer removed, and then relaxed for another $5$\,ps. The final configuration of this system was then used as the starting point of all subsequent simulations, allowing for consistency between them all. The system is then ran for up to $35$\,ps at temperatures ranging from $300$\,K to $2000$\,K. 

As it is impossible to precisely track a vacancy in a crytal, as it does not truly exist, Hu \al opt instead to measure the displacement of the vacancy's nearest neighbours in the surface, as vacancies move by exchanging positions with one of their neighbours. It is only necessary to measure the positions in the surface, as Halicioglu \al\cite{Halicioglu} previously determined that it is energetically unfavourable for a vacancy to diffuse deeper into the bulk. 

Hu \al found that full vacancy migration is only achieved at and above $1400$\,K, with simulations ran in the $1000 - 1300$\,K range showing only a partial relaxation of the surface neighbours into an intermediate position which they remain in until the end of the simulation. For $1400 - 1800$\,K, the surface neighbour relaxes to the intermediate position for some time, before finally moving all the way to the vacancy site, implying that the vacancy has fully migrated to the surface. Hu \al claim that this is the first time that the two-step migration phenomena has been observed, with teh intermediate vacancy position being much closer to the neighbour's original site than the vacancy site. For $2000$\,K the surface neighbour migrates to the vacancy site fully in one motion. These results differ to those seen in experiment, as mentioned in the paper, Davies \al\cite{Davies} have showed that in Type IIa diamond, the vacancy concentration greatly decreases after annealing at a temperature range of $973 - 1023$\,K. This would imply that the vacancy is fully mobile, as was seen in the simulations above $1400$\,K. This discrepancy is explained by Hu \al to be caused by how the temperatures are read: Davies \al are measuring the temeprature of the substrate on which the diamond is grown, however the temperature of the surface is likely to be much hotter. Another cause of the higher required migration temperature observed by Hu \al could be due to the use of the Tersoff potential, which is likely to overbind in cases like these, stopping the vacancy from migrating at the correct temperature (CITATION NEEDED).

\printbibliography

\end{document}