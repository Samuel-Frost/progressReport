\documentclass[10pt,a4paper,twocolumn,twoside]{extarticle}

\title{Progress Report}
\author{Samuel James Frost}
\def\ID{20184093}
\usepackage[utf8]{inputenc}
\usepackage[UKenglish]{babel}
\usepackage[UKenglish]{datetime}
\usepackage[T1]{fontenc}
\usepackage{geometry}
\usepackage{xspace}
\usepackage{varwidth}
\usepackage{svg}
\usepackage{tikz}
\usetikzlibrary{calc,math,arrows,arrows.meta,decorations.pathreplacing}
% \usepackage{pgfplots}
\usepackage{pgffor}
\usepackage{pgfmath}
\usepackage[super]{nth}
\usepackage{romannum}
\AtBeginDocument{\pagenumbering{arabic}}
\usepackage{float}
\usepackage{caption}
\usepackage{makecell}
\usepackage{tabularx}
\usepackage{multirow}
\renewcommand\tabularxcolumn[1]{m{#1}}
\usepackage{array}
\usepackage{amsmath,mleftright,amsthm,amsfonts,amssymb,amscd,nccmath}
\usepackage{centernot}
\usepackage{relsize}
\usepackage{physics}
\usepackage{polynom}
\usepackage{xparse}
\usepackage{fancyhdr}
\usepackage{titlesec}
\usepackage{accents}
\usepackage[scaled=1.15]{urwchancal}
\usepackage{bbding}
\usepackage{xfrac}
\usepackage{etoolbox}
\usepackage{siunitx}
\usepackage{parskip}
\usepackage{multicol}
\usepackage{enumerate}
\usepackage{enumitem}
\usepackage{mathtools}
\usepackage{mathrsfs}
\usepackage{lmodern}
\usepackage{slantsc}
\usepackage{bold-extra}
\usepackage{mfirstuc}
\usepackage{suffix}
\usepackage{csquotes}
\usepackage{esint}     % More integral types
\usepackage{extsizes}  % More fonts sizes
\usepackage{anyfontsize}
\usepackage{tocloft}   % Table of contents customisation

%% Font
% \usepackage{ebgaramond}
% \usepackage[cmintegrals,cmbraces]{newtxmath}
% \usepackage{ebgaramond-maths}
% \makeatletter
%   \DeclareSymbolFont{ntxletters}{OML}{ntxmi}{m}{it}
%   \SetSymbolFont{ntxletters}{bold}{OML}{ntxmi}{b}{it}
%   \re@DeclareMathSymbol{\leftharpoonup}{\mathrel}{ntxletters}{"28}
%   \re@DeclareMathSymbol{\leftharpoondown}{\mathrel}{ntxletters}{"29}
%   \re@DeclareMathSymbol{\rightharpoonup}{\mathrel}{ntxletters}{"2A}
%   \re@DeclareMathSymbol{\rightharpoondown}{\mathrel}{ntxletters}{"2B}
%   \re@DeclareMathSymbol{\triangleleft}{\mathbin}{ntxletters}{"2F}
%   \re@DeclareMathSymbol{\triangleright}{\mathbin}{ntxletters}{"2E}
%   \re@DeclareMathSymbol{\partial}{\mathord}{ntxletters}{"40}
%   \re@DeclareMathSymbol{\flat}{\mathord}{ntxletters}{"5B}
%   \re@DeclareMathSymbol{\natural}{\mathord}{ntxletters}{"5C}
%   \re@DeclareMathSymbol{\star}{\mathbin}{ntxletters}{"3F}
%   \re@DeclareMathSymbol{\smile}{\mathrel}{ntxletters}{"5E}
%   \re@DeclareMathSymbol{\frown}{\mathrel}{ntxletters}{"5F}
%   \re@DeclareMathSymbol{\sharp}{\mathord}{ntxletters}{"5D}
%   \re@DeclareMathAccent{\vec}{\mathord}{ntxletters}{"7E}
% \makeatother

%% Cal fonts
\DeclareMathAlphabet{\pazocal}{OMS}{zplm}{m}{n}
\newcommand\call[1]{\pazocal{#1}}
\newcommand\calll[1]{\mathcal{#1}}
\newcommand\callll[1]{\mathscr{#1}}

%% Nice empty-set.
\newcommand\oldemptyset\emptyset
\renewcommand\emptyset{\mathlarger{\mathlarger\varnothing}}


%% Theorems, definitions, remarks, lemmas, corrolaries, &c.
\theoremstyle{plain}
\newtheorem{theorem}{Theorem}[section]
\newtheorem*{theorem*}{Theorem}
\newtheorem{corollary}{Corollary}[theorem]
\newtheorem{lemma}{Lemma}[theorem]
\newtheorem{lemmaalone}{Lemma}[section]
\newtheorem{proposition}{Proposition}[theorem]
\newtheorem{problem}{Problem}[theorem]
\newtheorem{conjecture}{Conjecture}[theorem]
\newtheorem{claim}{Claim}[theorem]
\newtheorem*{claim*}{Claim}

\newtheorem{fact}{Fact}[theorem]
\newtheorem{assumption}{Assumption}

\theoremstyle{remark}
\newtheorem{construction}{Construction}[theorem]
\newtheorem{observation}{Observation}[theorem]

\theoremstyle{definition}
\newtheorem{definition}{Definition}[section]
\newtheorem{axiom}{Axiom}[definition]
\newtheorem{example}{Example}

\newtheoremstyle{nbremark}%
    {}{}{\normalfont}{}{\raisebox{-0.5mm}{\;{\Large\lefthand}\;}\itshape}%
    {.\ }{  }{}
\theoremstyle{nbremark}

\newtheorem*{remark*}{Remark}
\newtheorem{remark}{Remark}[theorem]
\newtheorem{defremark}{Remark}[definition]
\newtheorem{secremark}{Remark}[section]


%% Inline remark*
\newcommand\inlineremark[1]{\begin{remark}#1\end{remark}}
\WithSuffix\newcommand\inlineremark*[1]{\begin{remark*}#1\end{remark*}}

%% Theorem referencing
\makeatletter
\newcommand{\thref}[1]{\@splitref#1\@nil}
%\def\@splitref#1:#2\@nil{{\bfseries\small\capitalisewords{#1}~\ref{#1:#2}}}
\def\@splitref#1:#2\@nil{{{#1}~\ref{#1:#2}}}
\makeatother

%% Paragraph formattting.
\setlength{\parindent}{5ex}
\setlength{\parskip}{1ex}


%% Page geometry
\geometry{
    a4paper,
    hmarginratio=1:1,
    textwidth=150mm,
    top=35mm,
}

%% Header layout.
\renewcommand{\sectionmark}[1]{\markboth{#1}{}}
\fancyhf{}
\headheight 14pt
\fancyhead[RO]{\papertitle}
\fancyhead[LE]{\itshape\nouppercase{\leftmark}}
\fancyhead[RE,LO]{\thepage}
\pagestyle{fancy}

\makeatletter
\let\papertitle\@title
\let\paperauthor\@author
\makeatother


%% Section, subsection and section format
\titleformat{\section}[block]%
	{\fontsize{11}{10}}%
	{\rlap{{\large\S}\,\thesection.}}%
	{0pt}%
	{\scshape\hspace*{.05\columnwidth}\begin{minipage}[t]{.9\columnwidth}\centering}%
	[\end{minipage}\vspace{1pt}]

\titleformat{\subsection}[runin]%
	{}%
	{\S\,\thesubsection.}%
	{2ex}%
	{\bfseries}[.]

\titleformat{\subsubsection}[runin]%
	{}%
	{\S\,\thesubsubsection.}%
	{2ex}%
	{\bfseries}[.]

% ToC
\renewcommand{\cftaftertoctitle}{\hfill}
\renewcommand{\cftaftertoctitleskip}{-10pt}
\renewcommand{\cfttoctitlefont}{\hfill\fontsize{11}{10}\scshape}
\renewcommand{\cftsecfont}{\scshape}
\renewcommand{\cftsecleader}{\bfseries\ \cftdotfill{\cftdotsep}}
\renewcommand{\cftsecpresnum}{}
\renewcommand{\cftsecaftersnum}{.}
\renewcommand{\cftsecnumwidth}{4ex}
\renewcommand{\cftdot}{\ensuremath{\cdot}}
\renewcommand{\cftdotsep}{1}

%\usepackage[shortlabels]{enumitem}
\setlength{\labelsep}{1em}
\setlist{wide=0pt,leftmargin=*}

\DeclareFontFamily{OT1}{pzc}{}
\DeclareFontShape{OT1}{pzc}{m}{it}%
{<-> s * [1.15] pzcmi7t}{}
\DeclareMathAlphabet{\mathpzc}{OT1}{pzc}{m}{it}

%\mathtoolsset{showonlyrefs}
\newtagform{noparen}{(}{)}
\usetagform{noparen}
\renewcommand{\eqref}[1]{(\refeq{#1})}
\renewcommand{\theequation}{\arabic{section}.\arabic{equation}}

% Inner product
\DeclarePairedDelimiterX{\inp}[2]{\langle}{\rangle}{#1, #2}
% Floor and ceil
\DeclarePairedDelimiter\ceil{\lceil}{\rceil}
\DeclarePairedDelimiter\floor{\lfloor}{\rfloor}

\newcommand\bfit[1]{\textbf{\textit{#1}}}

\newcommand\avg[1]{\left\langle{#1}\right\rangle}
\mathchardef\Re="023C
\mathchardef\Im="023D
\let\oldRe\Re
\let\oldIm\Im
\renewcommand\Re[1]{\oldRe\mathfrak{e}\left\{#1\right\}}
\renewcommand\Im[1]{\oldIm\mathfrak{m}\left\{#1\right\}}
\newcommand\C{\mathbb{C}}
\newcommand\R{\mathbb{R}}
\newcommand\Q{\mathbb{Q}}
\newcommand\N{\mathbb{N}}
\newcommand\Z{\mathbb{Z}}
\newcommand\lhs{\text{L.H.S.}}
\newcommand\rhs{\text{R.H.S.}}
\newcommand\defeq{\coloneqq}
\newcommand*{\dif}[1]{\mathop{{\rm d}#1}}
\newcommand\et{{\;\textit{\&}\:}}
\newcommand\etc{\textit{\&\hspace{-0.7pt}c}.\@\xspace}
\newcommand\ie{\textit{i.\hspace{-1.2pt}e}.\@\xspace}
\newcommand\eg{\textit{e.\hspace{-1pt}g}.\@\xspace}
\newcommand*{\mf}{\mathfrak}
\newcommand{\bs}{\textbackslash}
\SetLabelAlign{parright}{\parbox[t]{\labelwidth}{\raggedleft#1}}
\newlist{questions}{enumerate}{3}
\setlist[questions]{itemsep=5mm,listparindent=\parindent}
\setlist[questions,1]{align=left,label={\arabic*.}}
\setlist[questions,2]{align=left,labelwidth=4ex,label={(\alph*)}}
\setlist[questions,3]{align=left,label={(\roman*)},labelwidth=7mm}
\newcommand\question[2]{\item #1\hfill[#2]}

\newcommand{\markshere}[1]{%
    \ifmmode\eqno\else\hspace*{\fill}\fi
    \textrm{[#1]}}

\newcommand\qitem[2][\relax]{\item #2{%
    \phantom{1pt}%
    \ifx\relax#1 \ \else \markshere{#1}\fi}}

\makeatletter
\newcommand{\skipitems}[1]{%
  \addtocounter{\@enumctr}{#1}}
\makeatother

\setlength{\jot}{10pt}

\let\abs\undefined
\let\norm\undefined
\DeclarePairedDelimiter\abs{\lvert}{\rvert}%
\DeclarePairedDelimiter\norm{\lVert}{\rVert}%

\makeatletter
\let\oldabs\abs
\def\abs{\@ifstar{\oldabs}{\oldabs*}}

\let\oldnorm\norm
\def\norm{\@ifstar{\oldnorm}{\oldnorm*}}
\makeatother

\newcommand{\infdiv}{D\infdivx}
\DeclarePairedDelimiter{\enorm}{\lVert}{\rVert}

\NewDocumentCommand{\evalat}{sO{\big}mm}{%
  \IfBooleanTF{#1}
   {\mleft. #3 \mright|_{#4}}
   {#3#2|_{#4}}%
}

\newcommand\m{\:\textrm{m}}
\newcommand\M{\:\Big[\textrm{m}\Big]}
\newcommand\mm{\:\textrm{mm}}
\newcommand\MM{\:\Big[\textrm{mm}\Big]}
\newcommand\un{\underline}
\newcommand\s{\:\textrm{s}}
\newcommand\bS{\:\Big[\textrm{S}\Big]}
\newcommand\ms{\:\frac{\textrm{m}}{\textrm{s}}}
\newcommand\MS{\:\Big[\frac{\textrm{m}}{\textrm{s}}\Big]}
\newcommand\mss{\:\frac{\textrm{m}}{\textrm{s}^2}}
\newcommand\MSS{\:\Big[\frac{\textrm{m}}{\textrm{s}^2}\Big]}

\makeatletter
\newcommand*\MY@leftharpoonupfill@{
    \arrowfill@\leftharpoonup\relbar\relbar
}
\newcommand*\MY@rightharpoonupfill@{
    \arrowfill@\relbar\relbar\rightharpoonup
}
\newcommand*\overleftharpoon{
    \mathpalette{\overarrow@\MY@leftharpoonupfill@}
}
\newcommand*\overrightharpoon{
    \mathpalette{\overarrow@\MY@rightharpoonupfill@}
}

\newcommand*\@dblsty@mathpalette[2]{
    \mathchoice
        {#1\displaystyle       \scriptstyle       {#2}}
        {#1\textstyle          \scriptstyle       {#2}}
        {#1\scriptstyle        \scriptscriptstyle {#2}}
        {#1\scriptscriptstyle  \scriptscriptstyle {#2}}
}
\newcommand*\@dblsty@overarrow@[4]{
    \vbox{\ialign{##\crcr
        #1#3\crcr
        \noalign{\nointerlineskip}
        \(\m@th\hfil #2#4\hfil\)\crcr
    }}
}
\newcommand*\smalloverleftharpoon{%
    \@dblsty@mathpalette{\@dblsty@overarrow@\MY@leftharpoonupfill@}%
}
\newcommand*\smalloverrightharpoon{%
    \@dblsty@mathpalette{\@dblsty@overarrow@\MY@rightharpoonupfill@}%
}
\makeatother
\newcommand{\poon}{\overrightharpoon}
\newcommand{\spoon}{\smalloverrightharpoon}
\newcommand\dom{\mathop{\rm dom}\nolimits}
\newcommand\ran{\mathop{\rm ran}\nolimits}
\newcommand\variance{\mathop{\rm var}\nolimits}
\newcommand\pmass[1]{\mathop{p_{\sub #1}}\nolimits}
\newcommand\cum[1]{\mathop{F_{\sub #1}}\nolimits}
\newcommand\Po{\mathop{\rm Po}\nolimits}
\newcommand{\set}[1]{\left\{\,#1\,\right\}}
\newcommand{\st}{\: : \:}
\newcommand{\Lim}[1]{\raisebox{0.5ex}{\scalebox{0.8}{$\displaystyle \lim_{#1}\;$}}}

\newcommand{\super}{\textsuperscript}
\renewcommand{\deg}{{\si{\degree}}}
\newcommand{\numero}{N\super{\underline{o}}}
\newcommand{\ihat}{\hat{{\imath}}}
\newcommand{\jhat}{\hat{{\jmath}}}
\newcommand{\khat}{\hat{k}}

\newcommand{\smallerrel}[1]{\mathrel{\mathpalette\smallerrelaux{#1}}}
\newcommand{\smallerrelaux}[2]{\raisebox{.1ex}{\scalebox{.75}{$#1#2$}}}

\newcommand{\shrink}{\smallerrel}
%% Defining operators
\let\Implies\implies
\let\Impliedby\impliedby
\let\Iff\iff
\renewcommand\implies{\;\mathop{\Rightarrow}\;}
\renewcommand\impliedby{\;\mathop{\Leftarrow}\;}
\renewcommand\iff{\;\mathop{\Leftrightarrow}\;}
\newcommand\mequiv\Leftrightarrow  %% Material equivalence.
\newcommand{\comp}{\mathbin{\shrink{\circ}}}
%\newcommand{\compl}[1]{{#1}^\complement}
\newcommand{\compl}[1]{\overline{#1}}
\newcommand\given{\mathop{|}}
\newcommand\forany\forall
\newcommand\forsome\exists
\newcommand\exactlyone{\exists!}
\newcommand\onlyone{\exists!}

\newcommand{\sub}{\mathchoice{}{}{\scriptscriptstyle}{}}

\newcommand{\then}{\Rightarrow\ }
\newcommand{\btw}[1]{\noalign{\centering\parbox{8cm}{#1}}}
\newcommand{\asside}[1]{\qquad\text{\small #1}}

\newcommand\mat[1]{\begin{bmatrix}#1\end{bmatrix}}
\newcommand\detmat[1]{\begin{vmatrix}#1\end{vmatrix}}

\usepackage{fourier-orns}

\usetikzlibrary{patterns}

\makeatletter
% \newcommand{\pgfplotsdrawaxis}{\pgfplots@draw@axis}
% \makeatother
% \pgfplotsset{only axis on top/.style={axis on top=false, after end axis/.code={
%              \pgfplotsset{axis line style=opaque, ticklabel style=opaque, tick style=opaque,
%                           grid=none}\pgfplotsdrawaxis}}}

\newcommand{\drawge}{-- (rel axis cs:1,0) -- (rel axis cs:1,1) -- (rel axis cs:0,1) \closedcycle}
\newcommand{\drawle}{-- (rel axis cs:1,1) -- (rel axis cs:1,0) -- (rel axis cs:0,0) \closedcycle}


\newcommand{\NB}{\raisebox{-1mm}{\,{\Large \lefthand}\ \,}}
%\newcommand{\NB}{\,{\bfseries N\hspace{-1.6mm}B}\ \,}

\newcommand{\spacer}[1]{%
    \vspace*{\fill}
    \hspace*{-\leftmargin}{\hfil\itshape#1\par}
    \vspace*{\fill}}

\renewcommand{\arraystretch}{1.2}

%%% TikZ Example %%%

% \begin{tikzpicture}
%     \begin{axis}[
%         axis lines = middle,
%         xlabel = $x$,
%         ylabel = $f(x)$,
%         ylabel near ticks,
%         grid = major,
%         xtick = {4.85, 3, -1, -1.85},
%         ytick = {5, -27},
%         ymax = 10,
%         ymin = -30,
%         xmax = 6,
%         width=13cm,
%         height=9cm
%     ]
%     %\addplot[domain=0:370]{}
%     \addplot [
%         domain=-3:8,
%         samples=200,
%         color=red,
%     ]
%     {x^3 - 3*x^2 - 9*x};
%     \addlegendentry{$x^3 - 3x^2 - 9x$}

%     \end{axis}
% \end{tikzpicture}

\newcommand{\kcal}{kcal mol\(^{-1}\)}
\usepackage{amsmath}
\usepackage[style=chem-rsc,citestyle=chem-rsc]{biblatex}
\addbibresource{references.bib}

\usepackage[font=small]{caption}
\usepackage{lipsum}
\usepackage{xcolor}
\newcommand{\ntvh}{N$_2$VH }
\newcommand{\al}{\emph{et al. }}
\newcommand{\oA}{\si{\angstrom}}
\renewcommand{\d}{\text{d}}
\title{Frist Year Progress Report}
\author{Samuel~J.~Frost}

\begin{document}
	\thispagestyle{empty}
	\twocolumn[
	\begin{@twocolumnfalse}
		\begin{center}
			\vspace*{-10mm}
			{\Large\scshape\papertitle}\\
			\vspace{2ex}
			{\itshape\paperauthor}
		\end{center}
		\centering\noindent\rule{0.9\textwidth}{0.4pt}
		\begin{abstract}
			Certainly! Here's an abstract for your first year PhD progress report!
		\end{abstract}
		\centering\noindent\rule{0.9\textwidth}{0.4pt}\\
		\vspace{1cm}
	\end{@twocolumnfalse}]
	\pagenumbering{arabic}
	\tableofcontents

	
\section{Introduction}
The final aim of my PhD is to be able to accurately model the effects of radiation damage in diamond over large time scales, taking in to account quantum mechanical effects.

\section{Reorientation of Hydrogen in N$_2$VH}
\subsection{Calculation}
The \ntvh defect consists of a vacancy (V) surrounded by two substitutional Nitrogens (N$_2$), with a Hydrogen bonded to one of the two remaining carbon atoms (H). C$_{2v}$ symmetry is reported from EPR runs in both the X ($8-12$\,GHz) and Q-band ($30-50$\,GHz) range \cite{Hartland}. This would suggest that the Hydrogen atom is reorientating between the Carbon atoms fast enough that its position is appearing as an averaged position of the two equivalent sites, giving rise to a higher order of symmetry. It would therefore be of interest to calculate the energy required for this reorientation to occur, so an estimate of the frequency of the reorientation can be made. 

A nudged elastic band (NEB) calculation was performed in order to find the energy barrier of the tunnelling path \cite{NEB}.
This method finds a minimum energy path (MEP) between two different states, in this case the Hydrogen moving between two equivalent carbons.
Firstly, a fully relaxed configuration of the two different states are required before any path optimisation can occur.
A diamond lattice was set up with $64$ atoms, one Carbon atom was removed, and two neighbours in the same plane were replaced with Nitrogen atoms, a Hydrogen atom was then placed near one of the two remaining carbon atoms. A geometry optimisation was then carried out in CASTEP, using the PBE functional, a plane wave cut off energy of $1\,000$\,eV, and an equally spaced Monkhurst-Pack grid of $(4 4 4)$. The system was optimised until no force was over $0.05$\,eV\,\AA$^{-1}$. This was then repeated for the other equivalent Carbon atom. Their energies are within $0.001$\,eV, showing equivalent sites. 
An initial 'guess' for the unoptimised path between the two systems is needed, for this, a simple a linear trajectory of the Hydrogen atom between the two Carbon atoms was devised. The trajectory contains an odd number of images, this is to ensure that it captures the saddle point of the path that is, due to the symmetry of the system, likely to be in the middle of the trajectory. 

The final, optimised, NEB path of the Hydrogen between the two equivalent carbon atoms, is seen in figure . The maximum energy found at the saddle point is $0.536$\,eV, with a full reaction path length of $1.1$\,\AA, and a full width half maximum (FWHM) of $0.3781$\,\AA. This differs from values found in literature, where the reported height and reaction path length are said to be $0.9$\,eV and $0.6$\,{\AA} respectively \cite{Peaker}. There are many differences between these two calculations, namely that \textcite{Peaker} uses a Gaussian basis set, whereas CASTEP uses a plane wave basis set, and the simulation cell is made up of $1\,000$ atoms, the increased number of atoms has the advantage of minimising finite size effects. 

\textcite{Peaker} takes the width of the barrier as the \emph{displacement} of the Hydrogen atom, the difference between the initial and final configuration. A higher displacement of $0.89$\,{\AA} was found here, however taking the raw displacement of the Hydrogen atom does not account for the path it takes during reorientation, nor does it fully utilise the minimum energy path that the NEB calculation found. A more appropriate approximation would be to take the fully optimised path of the Hydrogen atom as the width of the barrier, and so to map the potential energy barrier to the Hydrogen path, instead of the reaction coordinate. This is reasonable as the majority of the movement stems from the Hydrogen, with the carbon atoms slightly relaxing under bond breaking and forming. The full path of the Hydrogen atom is $1$\,{\AA}, only $0.1$\,{\AA} more than the full reaction coordinate. The potential barrier height was found to be almost half of that found in the literature, this could be due to a more optimised MEP, or a more accurate basis. The \emph{first} unoptimised MEP has a barrier height of $0.844$\,eV, much closer to that found in the literature. 
\subsection{Analysis}
With a barrier height and width determined it is now possible to calculate the probabilities of overcoming the barrier. As the barrier is a non-trivial shape, two different approximations will be made for different uses: approximating the barrier as a finite square potential, and the WKB potential. 

For the finite square potential, it is common to take the FWHM as the width, and the saddle point as the barrier height. The classical rate of reoritentation can be calculated as 
\begin{align}
    \Gamma &= A\exp({\frac{-E_a}{k_BT}})
\end{align}
where A is the attempt frequency, $E_a$ is the activation energy, taken to be the barrier height, $k_B$ is Boltzmann's constant, and $T$ is the temperature. The average attempt frequency was calculated in section to be $34.52$\,THz, . This would give a classical reorientation rate of $\Gamma = 34.171$\,KHz at room temperature. This is quite fast, but nowhere the rates seen in EPR, so the Hydrogen must be quantum tunnelling.

Taking the finite square potential approximation, the probability the Hydrogen atom tunnelling can be calculated as 
\begin{align}
    P &= \exp(\frac{-4a\pi}{h}\sqrt{2m(V-E)})
\end{align}

where $a$ is the width of the barrier, taken to be the FWHM, $m$ is the mass of the tunnelling particle, $V$ is the potential energy of the barrier, and $E$ is the energy of the particle. As an approximation the ground state energy of the Hydrogen atom can be taken as that of a simple harmonic oscillator 
\begin{align}
    E_0 = \frac{1}{2}h\nu
\end{align}
where $\nu$ is the frequency of the oscillations. The frequency of the hydrogen atom was , such that the ground state energy was thus . As EPR is typically performed at temperatures below $10$\,K, it is sensible to assume that the Hydrogen is in its ground state. 

A more accurate approximation of the nature of the potential barrier is the WKB approximation. This takes the form of 
\begin{align}
    P &= \exp(\frac{-4\pi}{h}\int_{a}^{b}\sqrt{2m(V(x)-E)}{\d}x)
\end{align}
where $a$ and $b$ are the \emph{turning points} of the barrier, such that $V(x) = E$. This approximation retains the shape of the barrier.


%Put this bit in later 
%This path is the reaction coordinate, so it accounts for the movement of every atom, however the majority of the movement stems from the Hydrogen, with the carbon atoms slightly relaxing under bond breaking and forming. This makes it possible to map the energy from the reaction coordinate, which has a length of $1.1$\,\AA to the Hydrogen atom's path, which has a length of $1$\,\AA. The small difference between these two values makes it a reasonable approximation to make.



\section{Vacancy Migration in a Pure Diamond Lattice}

There is much concern of how to remove vacancies from diamond, in order to create a perfect crystal. Annealing techniques are used to heat the diamond up to a certain temperature and cool it down, in hopes of making the vacancies turn mobile and rising to the surface, effectively eliminating themselves from the bulk. Previous experimental results have shown that this starts to occur at ~$800$\,K (CITATION NEEDED, NUMBER COULD BE WRONG), however it could also be useful to know the energies required for a vacancy to migrate, and also the energy barrier involved for the formation of a vacancy. Molecular dynamics simulations can provide an insight into how the mechanisms involved in vacancy migration, and to inform later experiments. Hu \al, as shown in section \ref{Hu}, have used a tersoff potential to model the vacancy migration of an atom from the second layer of a diamond ($001$) surface to the top layer. Their results showed that this process beigns to occur at $1400$\,K, deviating from known experimental results. Hu \al argue that is due to how the temperature of the diamond is measured in experiments, that the temperature of the surface is in fact much higher than measured, and so would align more with their findings. The results of Hu \al are quite old, however the methods used are not out of date, so it is reasonable to try and recreate similar results, and expand upon their work using more state-of-the-art techniques. 

The same system was set up as Hu \al described in section \ref{Hu}: a $1$\,fs timestep was used, as well as a $100$\,fs timestep for the velocity rescaling method, and a $1\,000$\,fs timestep for the pressure rescaling method. There is no mention in the original paper what the timestep for the velocity rescaling method is, nor is a pressure rescaling ensemble used at all. The choice to use the NPT ensemble came from the need to ensure that the crystal can expand slightly under higher temperatures. Following Hu \al, an initial system was created and relaxed at $300$\,K for $5$\,ps, before a vacancy was created in the second layer and it was allowed to relax again for another $5$\,ps. The final configuration of this system formed the starting configuration of all further simulations. The system was then allowed to run at temperatures in the $300$--$2\,000$\,K range. Taking inspiration from Hu \al, the positions of the atoms neighbouring the vacancy were extracted from the MD run. The atom with the lowest average distance to the vacancy site was taken to be the atom that was diffusing, which was then used for analysis. 

%Basically 1600 is when it really starts to migrate properly, we see two-step (maybe even three step) migration at 1600 and above, it does happen at 1400 but not always, probably a high chance of randomness -> monte carlo, at 2000 strange things happen because it is so hot, tersoff may be "breaking down". For GAP it seems to migrate at EVERY temperature, not good, we also see no migration from the third at any temperature for gap, also not good, could just be (001) surface however . Maybe explore NVT instead of NPT properly? -> makes no difference THEY USE CM^2 / s -> CHECK UNITS AND CONVERT


\section{Key Text Review}
\subsection{Hu \al}
\label{Hu}
"The Diffusion of Vacancies Near a Diamond ($001$) Surface" by Hu \al has played an important role in my research, influencing a large part of section . Hu \al used molecular dynamics to investigate vacancy diffusion in diamond surfaces at various temperatures, calculating the diffusion coefficient and diffusion barrier. Knowing the properties of vacancy defects, and at what temperatures they are mobile such that they might escape to the surface is important when dealing with synthetic diamodns. The paper is limited in that it only deals with vacancies found in the second layer of the ($001$) surface. Other surfaces, such as the cleavage ($111$) surface, also play important roles in experiment, not to mention vacancies that are found further into the bulk, such as in the third or fourth layer. The limited scope of this paper has influenced my further research found in section .

They construct their simulation as a unit cell repeated equally $5$ times in all $3$ cardinal directions, with periodic boundary conditions along the $x$ and $y$ axes, and the surfaces in the $z$ direction showing a ($001$) face. There is no mention of the boundaries of the cell in the $z$ direction, however as it is dealing with a surface diffusion, it is reasonable to assume that there is a sufficent vacuum gap, such that there are no external forces acting on the surface. The perfect diamond crystal is first allowed to relax for $5$\,ps at $300$\,K, before having an atom in the second layer removed, and then relaxed for another $5$\,ps. The final configuration of this system was then used as the starting point of all subsequent simulations, allowing for consistency between them all. The system is then ran for up to $35$\,ps at temperatures ranging from $300$--$2\,000$\,K. 

As it is impossible to precisely track a vacancy in a crytal, as it does not truly exist, Hu \al opt instead to measure the displacement of the vacancy's nearest neighbours in the surface, as vacancies move by exchanging positions with one of their neighbours. It is only necessary to measure the positions in the surface, as Halicioglu \al\cite{Halicioglu} previously determined that it is energetically unfavourable for a vacancy to diffuse deeper into the bulk. 

Hu \al found that full vacancy migration is only achieved at and above $1\,400$\,K, with simulations ran in the $1000$--$1\,300$\,K range showing only a partial relaxation of the surface neighbours into an intermediate position which they remain in until the end of the simulation. For $1\,400$--$1\,800$\,K, the surface neighbour relaxes to the intermediate position for some time, before finally moving all the way to the vacancy site, implying that the vacancy has fully migrated to the surface. Hu \al claim that this is the first time that the two-step migration phenomena has been observed, with teh intermediate vacancy position being much closer to the neighbour's original site than the vacancy site. For $2\,000$\,K the surface neighbour migrates to the vacancy site fully in one motion. These results differ to those seen in experiment, as mentioned in the paper, \textcite{Davies} have showed that in Type IIa diamond, the vacancy concentration greatly decreases after annealing at a temperature range of $973$--$1\,023$\,K. This would imply that the vacancy is fully mobile, as was seen in the simulations above $1\,400$\,K. This discrepancy is explained by Hu \al to be caused by how the temperatures are read: Davies \al are measuring the temeprature of the substrate on which the diamond is grown, however the temperature of the surface is likely to be much hotter. Another cause of the higher required migration temperature observed by Hu \al could be due to the use of the Tersoff potential, which is likely to overbind in cases like these, stopping the vacancy from migrating at the correct temperature (CITATION NEEDED).

\printbibliography

\end{document}